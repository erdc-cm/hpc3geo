\documentclass[12pt,dvips,letterpaper]{article}
\usepackage{epsfig,amsmath,amsfonts}
\begin{document}
%differential operators
\newcommand{\grad}{\nabla}
\newcommand{\deld}{\nabla \cdot}
\newcommand{\lap}{\Delta}
%boldface in math mode
\newcommand{\bm}[1]{\mbox{{\boldmath ${#1}$}}}
% vectors and tensors
\renewcommand{\vec}[1]{{\bf #1}}
\newcommand{\gvec}[1]{\mbox{{\boldmath ${#1}$}}}
\newcommand{\ten}[1]{\bar{\bm{#1}}}
%derivatives
\newcommand{\od}[2]{\frac{d {#1}}{d {#2}}}
\newcommand{\ods}[2]{\frac{d^2{#1}}{d {{#2}^2}}}
\newcommand{\pd}[2]{\frac{\partial {#1}}{\partial {#2}}}
\newcommand{\pds}[2]{\frac{\partial^2{#1}}{\partial {{#2}^2}}}
\newcommand{\pdsm}[3]{\frac{\partial^2{#1}}{\partial {#2}\,\partial {#3}}}
%funtional analysis
\newcommand{\abs}[1]{\left| #1 \right|}
\newcommand{\norm}[1]{\left\| #1 \right\|}
\newcommand{\iprod}[2]{\left( #1, #2 \right)}
\newcommand{\dprod}[2]{\left\langle #1, #2 \right\rangle}
%real numbers
\newcommand{\field}[1]{\mathbb{#1}}
\newcommand{\R}{\field{R}}
%funciton spaces
\newcommand{\M}{\mathcal{M}}
%delimiters
\newcommand{\pl}{\left(}
\newcommand{\pr}{\right)}
\newcommand{\sbl}{\left[}
\newcommand{\sbr}{\right]}
\newcommand{\dbl}{\left[\hspace{-0.05cm}\left[}
\newcommand{\dbr}{\right]\hspace{-0.05cm}\right]}
\newcommand{\cbl}{\left\{ }
\newcommand{\cbr}{\right\} }
\newcommand{\eqn}[1]{equation \ref {eq:#1}} 
\newcommand{\Eqn}[1]{Equation \ref {eq:#1}} 
\newcommand{\eqnst}[2]{equations \ref{eq:#1} and \ref{eq:#2}} 
\newcommand{\Eqnst}[2]{Equations \ref{eq:#1} and \ref{eq:#2}} 
\newcommand{\eqns}[2]{equations \ref{eq:#1}--\ref{eq:#2}} 
\newcommand{\Eqns}[2]{Equations \ref{eq:#1}--\ref{eq:#2}}
\newcommand{\msection}[1]{ \vspace{.2in} {\noindent \bf #1}.}
\renewcommand{\for}{\mbox{for}\quad}
%\newcommand{\for}{\mbox{for}\quad}
\newcommand{\argmin}{\mbox{argmin}}
\newcommand{\argmax}{\mbox{argmax}}
\newcommand{\fig}[1]{figure \ref{fig:#1}} 
\newcommand{\Fig}[1]{Figure \ref{fig:#1}} 
\newcommand{\figst}[2]{figures \ref {fig:#1} and \ref {fig:#2}} 
\newcommand{\Figst}[2]{Figures \ref {fig:#1} and \ref {fig:#2}} 
\newcommand{\figs}[2]{figures \ref{fig:#1}--\ref{fig:#2}} 
\newcommand{\Figs}[2]{Figures \ref{fig:#1}--\ref{fig:#2}}
\newcommand{\tab}[1]{table \ref {tab:#1}} 
\newcommand{\Tab}[1]{Table \ref {tab:#1}} 
\newcommand{\tabst}[2]{tables \ref {tab:#1} and \ref {tab:#2}} 
\newcommand{\Tabst}[2]{Tables \ref {tab:#1} and \ref {tab:#2}} 
\newcommand{\tabs}[2]{tables \ref{tab:#1}--\ref{tab:#2}} 
\newcommand{\Tabs}[2]{Tables \ref{tab:#1}--\ref{tab:#2}}
\newtheorem{theorem}{Theorem}
\newenvironment{neqnarray}[1]{\begin{minipage}[t]{6.5in}  \begin{minipage}[b]{1.0in} #1 \end{minipage}  \begin{minipage}[b]{5.5in}\begin{eqnarray}}{\end{eqnarray}\end{minipage}\end{minipage}}
\newcommand{\bneqnarray}[2]{\\ \\ \fbox{\begin{neqnarray}{#1} #2 \end{neqnarray}}\\ \\ \noindent} 

\newcommand{\R}{\mathbb{R}}
\newcommand{\framedEquation}[1]{\fbox{$ \begin{array}{rcl} #1 \end{array} $}}
\newcommand{\framedArray}[1]{\fbox{$ \begin{array}{rcl} #1 \end{array} $}}

\begin{center}
 The Richards Equation Model of Air and Water Flow in Porous Media
\end{center}
\begin{center} 
  C. E. Kees \\
  Center for Research in Scientific Computation \\
  Department of Mathematics \\
  North Carolina State University \\
  Raleigh, NC 27695-8205 \\
  (chris\_kees@ncsu.edu) \\
\end{center}
\begin{abstract}
  I present a brief model formulation for Richards' equation, a model
  of water flow in variably saturated porous media and a simple 1D
  test problem.\end{abstract}
\begin{center}
  Updated: October 1, 2003 \\
\end{center}

\section{Richards Equation}

The model can be written concisely as
\begin{equation}
\framedEquation{
\pd{[m(u)]}{t} + \deld \left[ - a( u) \grad u + \vec b(u) \right] = 0 }
\end{equation}
We still need to specify the nonlinearities. First consider an
alternate formulation.

\subsection{Kirchof Transformation}

We can rewrite the term  $a(u) \grad u$ as $\grad \phi$ where 
\begin{equation*}
\phi = \int_0^u a(v) dv
\end{equation*}
this trick is called the Kirchof transformation and $\phi(u)$
the Kirchof potential. If $\phi(u)$ is invertible then we can use
$\phi$ as the solution variable and there will be no nonlinearity in
the second order term. For instance, Richards' equation becomes
\begin{equation}
\framedEquation{
\pd{[m(\phi)]}{t} + \grad^2 \phi +  \deld \vec b(\phi)  = 0 
}
\end{equation}

\subsection{Nonlinearities}

We use the following algebraic relations to close the equation.
\begin{equation}
\boxedArray{
m(u) &=&  (1 + u^n)^(1/n - 1) \\
a(u) &=& k_s (
\end{equation}
\end{document}


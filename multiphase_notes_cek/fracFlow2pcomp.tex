\documentclass[12pt,dvips]{report}
\setcounter{secnumdepth}{5}
\usepackage{epsfig}
%... include macros 
%differential operators
\newcommand{\grad}{\nabla}
\newcommand{\deld}{\nabla \cdot}
\newcommand{\lap}{\Delta}
%boldface in math mode
\newcommand{\bm}[1]{\mbox{{\boldmath ${#1}$}}}
% vectors and tensors
\renewcommand{\vec}[1]{{\bf #1}}
\newcommand{\gvec}[1]{\mbox{{\boldmath ${#1}$}}}
\newcommand{\ten}[1]{\bar{\bm{#1}}}
%derivatives
\newcommand{\od}[2]{\frac{d {#1}}{d {#2}}}
\newcommand{\ods}[2]{\frac{d^2{#1}}{d {{#2}^2}}}
\newcommand{\pd}[2]{\frac{\partial {#1}}{\partial {#2}}}
\newcommand{\pds}[2]{\frac{\partial^2{#1}}{\partial {{#2}^2}}}
\newcommand{\pdsm}[3]{\frac{\partial^2{#1}}{\partial {#2}\,\partial {#3}}}
%funtional analysis
\newcommand{\abs}[1]{\left| #1 \right|}
\newcommand{\norm}[1]{\left\| #1 \right\|}
\newcommand{\iprod}[2]{\left( #1, #2 \right)}
\newcommand{\dprod}[2]{\left\langle #1, #2 \right\rangle}
%real numbers
\newcommand{\field}[1]{\mathbb{#1}}
\newcommand{\R}{\field{R}}
%funciton spaces
\newcommand{\M}{\mathcal{M}}
%delimiters
\newcommand{\pl}{\left(}
\newcommand{\pr}{\right)}
\newcommand{\sbl}{\left[}
\newcommand{\sbr}{\right]}
\newcommand{\dbl}{\left[\hspace{-0.05cm}\left[}
\newcommand{\dbr}{\right]\hspace{-0.05cm}\right]}
\newcommand{\cbl}{\left\{ }
\newcommand{\cbr}{\right\} }
\newcommand{\eqn}[1]{equation \ref {eq:#1}} 
\newcommand{\Eqn}[1]{Equation \ref {eq:#1}} 
\newcommand{\eqnst}[2]{equations \ref{eq:#1} and \ref{eq:#2}} 
\newcommand{\Eqnst}[2]{Equations \ref{eq:#1} and \ref{eq:#2}} 
\newcommand{\eqns}[2]{equations \ref{eq:#1}--\ref{eq:#2}} 
\newcommand{\Eqns}[2]{Equations \ref{eq:#1}--\ref{eq:#2}}
\newcommand{\msection}[1]{ \vspace{.2in} {\noindent \bf #1}.}
\renewcommand{\for}{\mbox{for}\quad}
%\newcommand{\for}{\mbox{for}\quad}
\newcommand{\argmin}{\mbox{argmin}}
\newcommand{\argmax}{\mbox{argmax}}
\newcommand{\fig}[1]{figure \ref{fig:#1}} 
\newcommand{\Fig}[1]{Figure \ref{fig:#1}} 
\newcommand{\figst}[2]{figures \ref {fig:#1} and \ref {fig:#2}} 
\newcommand{\Figst}[2]{Figures \ref {fig:#1} and \ref {fig:#2}} 
\newcommand{\figs}[2]{figures \ref{fig:#1}--\ref{fig:#2}} 
\newcommand{\Figs}[2]{Figures \ref{fig:#1}--\ref{fig:#2}}
\newcommand{\tab}[1]{table \ref {tab:#1}} 
\newcommand{\Tab}[1]{Table \ref {tab:#1}} 
\newcommand{\tabst}[2]{tables \ref {tab:#1} and \ref {tab:#2}} 
\newcommand{\Tabst}[2]{Tables \ref {tab:#1} and \ref {tab:#2}} 
\newcommand{\tabs}[2]{tables \ref{tab:#1}--\ref{tab:#2}} 
\newcommand{\Tabs}[2]{Tables \ref{tab:#1}--\ref{tab:#2}}
\newtheorem{theorem}{Theorem}
\newenvironment{neqnarray}[1]{\begin{minipage}[t]{6.5in}  \begin{minipage}[b]{1.0in} #1 \end{minipage}  \begin{minipage}[b]{5.5in}\begin{eqnarray}}{\end{eqnarray}\end{minipage}\end{minipage}}
\newcommand{\bneqnarray}[2]{\\ \\ \fbox{\begin{neqnarray}{#1} #2 \end{neqnarray}}\\ \\ \noindent} 

\begin{document}           % End of preamble and beginning of text.
\begin{center} \bf Compressible Two-Phase Flow Formulation \end{center}

For a homogeneous medium and compressible fluids, the equations
for two-phase flow in fractional flow form are
\begin{eqnarray}
 \pd{(\omega \rho_w S_w)}{t}  &=& - \deld [ \frac{\rho_n k_{rn}}{\mu_n} f_w \pd{p_c}{S_w} \grad S_w \nonumber \\
&&- \frac{\rho_n  k_{rn}}{\mu_n} f_w (\rho_n - \rho_w) \vec g 
 + f_w \vec q_t ]  \label{mass_balance} \\
\pd{ \{ \omega [ \rho_w S_w - \rho_n (1-S_w)] \} }{t} &=& -\deld  \vec q_t  \label{div_term} \\
\vec q_t &=& - K \l[ \chi \grad p_t  - \rho_t  \vec g \r] \\ 
K &=& \frac{\rho_w k_{rw}}{\mu_w}+ \frac{\rho_n k_{rn}}{\mu_w} \\
f_w &=& \frac{\frac{\rho_w k_{rw}}{\mu_w}}{\frac{\rho_w k_{rw}}{\mu_w}+ \frac{\rho_n k_{rn}}{\mu_w}}\\
p_t &=& \frac{p_n + p_w}{2} + \int_{1}^{S_w}  (\frac{1}{2} - f_w(s,p_t)) \pd{p_c}{S_w}(s) ds \\
\chi & = & 1 - \pd{}{p_t} \int_{1}^{S_w}  (\frac{1}{2} - f_w(s)) \pd{p_c}{S_w}(s) ds \\
\rho_t &=& ((1-f_w)\rho_n + f_w \rho_w) \\
\rho_w &=& \hat{\rho_w}(p_t)/\hat{\rho_w}(p_{t0})\\
\rho_n &=& \hat{\rho_n}(p_t)/\hat{\rho_n}(p_{t0})\\
\end{eqnarray}
The unknowns are $S_w$ and $p_t$. The first three equations are the
PDE's. The rest are nonlinearities. The $p-s-k$ functions and
densities still have to be specified. There are a couple of things to
note. 1) $p_t$ appears in its own definition so it is defined
implicitly. (Chavent and Jaffre prove the definition of $p_t$ is a
contraction, therefore $p_t$ is unique) 2) A new term $\chi$ shows up
in the pressure equation. This is because the ``Kirchof trick'' is
complicated by the fractional flow $f_w$ now depending on $S_w$ and $p_t$.
3) Given $p_t$ and $S_w$ you should be able to compute $p_w$ and $p_n$
independently so it shouldn't be strictly necessary to evaluate the
densities at $p_t$, but this is how Chavent and Jaffre do it so there
may be a deeper reason for doing it this way. As with the
incompressible version you should be able to plug all of the equations
into eqn \ref{mass_balance} and \ref{div_term} and recover the
equation for the wetting phase in the standard two-phase approach and
the sum of the wetting and nonwetting phase equations.I've lumped the
intrinsic permeability into $k$ since the medium is homogeneous but
otherwise this model should be identical to Chavent and Jaffre's
starting on pp199-202. With appropriate defintions in terms of the
above we get (note I've changed the sign of D from the last set of
notes just to make it consistent with standard advection diffusion
equations).
\begin{eqnarray}
\pd{M_w}{t} + \deld \l[ -D_w(S_w,p_t) \grad S_w + F_w(S_w,p_t) \r] &=&0 \\
\pd{M_t}{t} + \deld \l[ -D_t(S_w,p_t) \grad p_t+ F_t(S_w,p_t) \r] &=&0 \\
M_w - \hat{M_w}(S_w,p_t) &=& 0 \\
M_t - \hat{M_t}(S_w,p_t) &=& 0
\end{eqnarray}
If this were the incompressible case $p_t$ wouldn't show up in the
coefficients of the first and third equations so you could use an
explict method on those equations and an implicit method on the second
and fourth equations without any splitting error (that's IMPES). In
the compressible case we either have to do operator splitting or do
the whole thing implicitly. If at least part of both equations are
solved implicitly (say the diffusion part of the saturation equation
coupled with the entire pressure equation) then we could still only
invert diagonal and tridiagonal matrices if we're willing to back off
of Newton's method some. 

\end{document}









